\documentclass[10pt,a4paper]{article}
\usepackage{times} % ou seulement l'un, ou l'autre, ou times etc.
\usepackage[utf8]{inputenc}
\usepackage[T1]{fontenc}
\usepackage[french]{babel}
\usepackage{fancyhdr}
\usepackage{xcolor}
\usepackage{mathbbol}
\usepackage{geometry}
\usepackage[Algortihme]{algorithm}
\usepackage{algorithmic}
\usepackage{textcomp}
\usepackage{amsthm}
\usepackage{amsmath}
\usepackage{stmaryrd} 
\usepackage{listings}
\usepackage{tikz}
\usepackage{framed}
\usepackage{tcolorbox}
\usepackage{amssymb}
\usepackage{lmodern}


\newtheorem{lemme}{Lemme}[section]
\newtheorem{prop}{Propriété}[section]
\newtheorem*{dem}{Preuve}
\newtheorem{thm}{Théorème}[section]
\newtheorem{defe}{Définition}[section]

\newcommand{\Mod}[1]{ (\mathrm{mod}\ #1)}
\newcommand{\inte}[1]{\llbracket #1 \rrbracket}
\newcommand{\sur}[1]{\overline{#1}}

\DeclareMathOperator{\Ima}{Im}

\author{Alexandre}
\date{2023/2024}
\geometry{a4paper, top=5cm, bottom=3cm, left=3cm, right=3cm, hmargin=1.3cm,vmargin=3cm}
\definecolor{darkWhite}{rgb}{0.94,0.94,0.94}
\title{Description matricielle des nombres algébriques}


\begin{document}


\maketitle

\section{Avant-Propos}

Ce document a pour objet de donner quelques preuves de propriétés importantes en théorie algébrique des nombres.

\section{Introduction}

\begin{defe}
	Pour $n\in \mathbb{N}^*$ et $K$ un corps, on notera $\mathbf{M}_n(K)$ l'ensemble des matrices à coeffiecients dans $K$.
\end{defe}


\begin{defe}
Soit $x \in \mathbb{C}$. On dit que $x$ est \textit{algébrique} s'il est annulé par un polynôme à coefficients dans $\mathbb{Q}$. 
On note $\bar{\mathbb{Q}}$ l'ensemble des nombres algébriques.
\end{defe}
\begin{defe}
Si de plus $x$ est annulé par un polynôme unitaire à coeffecients entiers, on dit que $x$ est un entier algébrique. 
On note $\bar{\mathbb{Z}}$ l'ensemble des entiers algébriques. 
\end{defe}

\begin{defe}
	Pour $A$ et $B$ deux matrices de $\mathbf{M}_n(\mathbb{C})$, on définit le produit tensoriel :
	$$ 
A \otimes B =\left[\begin{array}{cccc}
a_{1,1} B & a_{1,2} B & \cdots & a_{1,n} B \\
a_{2,1} B & a_{2,2} B & \cdots & a_{2,n} B \\
\vdots & \vdots & \ddots & \vdots \\
a_{n,1} B & a_{n,2} B & \cdots & a_{n,n} B
\end{array}\right]
$$
	
\end{defe}



\begin{thm}
$\bar{\mathbb{Q}}$ est un sous-corps de $\mathbb{C}$
\end{thm}

\begin{dem}
$\bar{\mathbb{Q}}$ est bien inclus dans $\mathbb{C}$ et n'est pas vide. Soit $x$ et $y$ deux nombres algébriques. Soit $P$ et $Q$ deux polynômes annulateurs annulant respectivement $x$ et $y$. Soit $A$ et $B$ les matrices compagnons respectivement de $P$ et de $Q$. 
\end{dem}



	





\end{document}










